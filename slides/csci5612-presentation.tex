% Options for packages loaded elsewhere
\PassOptionsToPackage{unicode}{hyperref}
\PassOptionsToPackage{hyphens}{url}
%
\documentclass[
  ignorenonframetext,
]{beamer}
\usepackage{pgfpages}
\setbeamertemplate{caption}[numbered]
\setbeamertemplate{caption label separator}{: }
\setbeamercolor{caption name}{fg=normal text.fg}
\beamertemplatenavigationsymbolsempty
% Prevent slide breaks in the middle of a paragraph
\widowpenalties 1 10000
\raggedbottom
\setbeamertemplate{part page}{
  \centering
  \begin{beamercolorbox}[sep=16pt,center]{part title}
    \usebeamerfont{part title}\insertpart\par
  \end{beamercolorbox}
}
\setbeamertemplate{section page}{
  \centering
  \begin{beamercolorbox}[sep=12pt,center]{section title}
    \usebeamerfont{section title}\insertsection\par
  \end{beamercolorbox}
}
\setbeamertemplate{subsection page}{
  \centering
  \begin{beamercolorbox}[sep=8pt,center]{subsection title}
    \usebeamerfont{subsection title}\insertsubsection\par
  \end{beamercolorbox}
}
\AtBeginPart{
  \frame{\partpage}
}
\AtBeginSection{
  \ifbibliography
  \else
    \frame{\sectionpage}
  \fi
}
\AtBeginSubsection{
  \frame{\subsectionpage}
}
\usepackage{amsmath,amssymb}
\usepackage{iftex}
\ifPDFTeX
  \usepackage[T1]{fontenc}
  \usepackage[utf8]{inputenc}
  \usepackage{textcomp} % provide euro and other symbols
\else % if luatex or xetex
  \usepackage{unicode-math} % this also loads fontspec
  \defaultfontfeatures{Scale=MatchLowercase}
  \defaultfontfeatures[\rmfamily]{Ligatures=TeX,Scale=1}
\fi
\usepackage{lmodern}
\usetheme[]{Madrid}
\ifPDFTeX\else
  % xetex/luatex font selection
\fi
% Use upquote if available, for straight quotes in verbatim environments
\IfFileExists{upquote.sty}{\usepackage{upquote}}{}
\IfFileExists{microtype.sty}{% use microtype if available
  \usepackage[]{microtype}
  \UseMicrotypeSet[protrusion]{basicmath} % disable protrusion for tt fonts
}{}
\makeatletter
\@ifundefined{KOMAClassName}{% if non-KOMA class
  \IfFileExists{parskip.sty}{%
    \usepackage{parskip}
  }{% else
    \setlength{\parindent}{0pt}
    \setlength{\parskip}{6pt plus 2pt minus 1pt}}
}{% if KOMA class
  \KOMAoptions{parskip=half}}
\makeatother
\usepackage{xcolor}
\newif\ifbibliography
\setlength{\emergencystretch}{3em} % prevent overfull lines
\providecommand{\tightlist}{%
  \setlength{\itemsep}{0pt}\setlength{\parskip}{0pt}}
\setcounter{secnumdepth}{-\maxdimen} % remove section numbering
% definitions for citeproc citations
\NewDocumentCommand\citeproctext{}{}
\NewDocumentCommand\citeproc{mm}{%
  \begingroup\def\citeproctext{#2}\cite{#1}\endgroup}
\makeatletter
 % allow citations to break across lines
 \let\@cite@ofmt\@firstofone
 % avoid brackets around text for \cite:
 \def\@biblabel#1{}
 \def\@cite#1#2{{#1\if@tempswa , #2\fi}}
\makeatother
\newlength{\cslhangindent}
\setlength{\cslhangindent}{1.5em}
\newlength{\csllabelwidth}
\setlength{\csllabelwidth}{3em}
\newenvironment{CSLReferences}[2] % #1 hanging-indent, #2 entry-spacing
 {\begin{list}{}{%
  \setlength{\itemindent}{0pt}
  \setlength{\leftmargin}{0pt}
  \setlength{\parsep}{0pt}
  % turn on hanging indent if param 1 is 1
  \ifodd #1
   \setlength{\leftmargin}{\cslhangindent}
   \setlength{\itemindent}{-1\cslhangindent}
  \fi
  % set entry spacing
  \setlength{\itemsep}{#2\baselineskip}}}
 {\end{list}}
\usepackage{calc}
\newcommand{\CSLBlock}[1]{\hfill\break\parbox[t]{\linewidth}{\strut\ignorespaces#1\strut}}
\newcommand{\CSLLeftMargin}[1]{\parbox[t]{\csllabelwidth}{\strut#1\strut}}
\newcommand{\CSLRightInline}[1]{\parbox[t]{\linewidth - \csllabelwidth}{\strut#1\strut}}
\newcommand{\CSLIndent}[1]{\hspace{\cslhangindent}#1}
\setbeamertemplate{navigation symbols}{}

\AtBeginDocument{
	\title[U.S. Inflation–Unemployment Dynamics]{The Dynamics Between Inflation and Unemployment: Empirical Evidence from U.S. Macroeconomic Data}
	\author[Taihei Sone]{Taihei Sone\inst{1}}
	\institute[]{\inst{1} College of Engineering and Applied Science\\University of Colorado Boulder\\Boulder, CO, USA\\Taihei.Sone@colorado.edu}
}

\setbeamerfont{caption}{size=\footnotesize}
\usepackage{float}
\usepackage{booktabs}
\usepackage{graphicx}
\newcommand\blfootnote[1]{%
	\begingroup
	\renewcommand\thefootnote{}\footnote{#1}%
	\addtocounter{footnote}{-1}%
	\endgroup
}

\usepackage{array}
\newcolumntype{L}{>{\centering\arraybackslash}m{3cm}}

\usepackage[table]{xcolor}
\usepackage{colortbl}
\definecolor{rowred}{RGB}{255, 200, 200}

\usepackage{arydshln}

\usepackage{tabularx}
\usepackage{bookmark}
\IfFileExists{xurl.sty}{\usepackage{xurl}}{} % add URL line breaks if available
\urlstyle{same}
\hypersetup{
  pdftitle={The Dynamics Between Inflation and Unemployment: Empirical Evidence from U.S. Macroeconomic Data},
  pdfauthor={Taihei Sone},
  hidelinks,
  pdfcreator={LaTeX via pandoc}}

\title{The Dynamics Between Inflation and Unemployment: Empirical
Evidence from U.S. Macroeconomic Data}
\author{Taihei Sone\inst{1}}
\date{2025-12-05}
\institute{\inst{1} College of Engineering and Applied
Science\textbackslash{} University of Colorado Boulder\textbackslash{}
Boulder, CO, USA\textbackslash{}
\href{mailto:Taihei.Sone@colorado.edu}{\nolinkurl{Taihei.Sone@colorado.edu}}}

\begin{document}
\frame{\titlepage}

\begin{frame}{Outline}
\phantomsection\label{outline}
\tableofcontents
\end{frame}

\section{Introduction}\label{introduction}

\begin{frame}{Introduction}
\phantomsection\label{introduction-1}
\begin{itemize}
\tightlist
\item
  The inflation--unemployment relationship (Phillips curve) is
  \textbf{unstable} and varies across economic regimes.\\
\item
  Recent evidence shows a \textbf{flattened and nonlinear} relationship,
  challenging conventional interpretations.\\
\item
  Understanding inflation--unemployment dynamics requires identifying
  \textbf{underlying macroeconomic factors} and \textbf{regime
  shifts}.\\
\item
  Machine learning methods offer useful tools for \textbf{classifying
  economic states}, though interpretability remains limited.
\end{itemize}

\begin{block}{This Study}
\phantomsection\label{this-study}
\begin{itemize}
\tightlist
\item
  Identify macroeconomic factors driving inflation and unemployment (via
  PCA)\\
\item
  Detect regime shifts using clustering methods\\
\item
  Evaluate regime identification using machine-learning classifiers
  (Naïve Bayes, Decision Trees, XGBoost)
\end{itemize}
\end{block}
\end{frame}

\section{Related Literature}\label{related-literature}

\begin{frame}{Related Literature}
\phantomsection\label{related-literature-1}
\begin{itemize}
\tightlist
\item
  Evidence on the inflation--unemployment relationship: trade-off,
  instability, and \textbf{flattening} over time

  \begin{itemize}
  \tightlist
  \item
    \tiny Phillips (1958); Samuelson and Solow (1960); Muth (1961);
    Lucas (1976); Mazumder and Ball (2011); Blanchard (2016)
  \end{itemize}
\item
  Phillips curve dynamics are \textbf{nonlinear and regime-dependent}

  \begin{itemize}
  \tightlist
  \item
    \tiny Hazell et al. (2022)
  \end{itemize}
\item
  Machine learning methods can help \textbf{identify economic states}
  and capture nonlinear patterns

  \begin{itemize}
  \tightlist
  \item
    \tiny Medeiros et al. (2021); Gogas, Papadimitriou, and Sofianos
    (2022)
  \end{itemize}
\end{itemize}

\textbf{Key Insight:}\\
The inflation--unemployment relationship is \textbf{not stable}; it
varies with economic regimes and underlying shocks.

\textbf{Research Gap:}\\
Existing studies address inflation, unemployment, regimes, or ML
\textbf{separately}, but do not provide a \textbf{unified, reproducible
framework} that\\
\quad\quad - jointly analyzes both variables,\\
\quad\quad - identifies economic regimes, and\\
\quad\quad - evaluates regime classification with ML.
\end{frame}

\section{Data}\label{data}

\begin{frame}{Data}
\phantomsection\label{data-1}
\begin{center}
\scriptsize
\renewcommand{\arraystretch}{0.8}
\begin{tabular}{p{2.7cm} p{3.5cm} p{0.5cm} p{1.5cm} p{1.5cm}}
\hline
\textbf{Category} & \textbf{Variable} & \textbf{Freq.} & \textbf{Source} & \textbf{Code} \\
\hline
Inflation & Core PCE & M & FRED & PCEPILFE \\
Unemployment & Unemployment Rate & M & FRED & UNRATE \\
 & Noncyclical Unemployment & Q & FRED & NROU \\
Regimes & Recession indicator & M & FRED & USREC \\
 & Zero Lower Bound dummy & M & FRED & - \\
 & COVID-19 period dummy & M & Created & - \\
Demand & Real GDP & Q & FRED & GDPC1\\
 & Real Potential GDP & Q & FRED & GDPPOT \\
 & Industrial Production & M & FRED & INDPRO \\
 & Retail Sales & M & FRED & RSAFS \\
Supply & Crude Oil Prices & M & FRED & MCOILWTICO \\
 & Import Price Index & M & FRED & IR \\
 & Labor Productivity & Q & FRED & OPHNFB \\
Labor Markets & Avg Hourly Earnings & M & FRED & CES050... \\
 & Labor Force Participation & M & FRED & CIVPART \\
 & Job Openings & M & FRED & JTSJOL \\
Monetary Policy & Fed Funds Rate & M & FRED & FEDFUNDS \\
 & Money Supply M2 & M & FRED & M2SL \\
 & Fed Total Assets & W & FRED & WALCL \\
Inflation Expectations & 5Y Breakeven & D & FRED & T5YIE \\
 & 10Y Breakeven & D & FRED & T10YIE \\
 & 1Y Exp. Inflation & M & UM Survey & PX\_MD \\
 & 5Y Exp. Inflation & M & UM Survey & PX5\_MD \\
\hline
\end{tabular}
\end{center}

\tiny Full list includes 23 series retrieved via FRED API or University
of Michigan CSV downloads.
\end{frame}

\section{Method}\label{method}

\begin{frame}{Method}
\phantomsection\label{method-1}
\begin{itemize}
\item
  \textbf{PCA (Principal Component Analysis)}\\
  Reduce dimensionality and extract latent inflation and business-cycle
  factors.
\item
  \textbf{K-Means \& Hierarchical Clustering}\\
  Identify macroeconomic regimes using unsupervised grouping.
\item
  \textbf{Naïve Bayes}\\
  Simple probabilistic classifier for high-inflation / high-unemployment
  regimes.
\item
  \textbf{Decision Trees}\\
  Nonlinear, interpretable classification based on threshold rules.
\item
  \textbf{Boosting (XGBoost)}\\
  Ensemble method capturing complex patterns and improving predictive
  accuracy.
\end{itemize}
\end{frame}

\section{Results \& Discussions}\label{results-discussions}

\begin{frame}{PCA Results: Macroeconomic Factor Structure}
\phantomsection\label{pca-results-macroeconomic-factor-structure}
\begin{columns}

  % Left side (figure)
  \begin{column}{0.48\textwidth}
    \begin{center}
      \includegraphics[height=0.85\textheight]{../results/3_PCA/pca_corr_heatmap.png}
    \end{center}
  \end{column}

  % Right side (text)
  \begin{column}{0.52\textwidth}
    \begin{itemize}
      \item \textbf{PC1: Inflation Factor}  
            Strong link to Core PCE \& expectations;  
            negative with unemployment.
    
      \item \textbf{PC2–PC3: Business Cycle}  
            Driven by production, sales,  
            and labor-market indicators.
    
      \item \textbf{Policy Signals}  
            Monetary Policy-related variables (FEDFUNDS, M2, and WALCL) spread across PCs.
    
      \item \textbf{Summary}  
            PCA extracts \textbf{inflation} vs. \textbf{cycle} factors.
    \end{itemize}
  \end{column}

\end{columns}
\end{frame}

\begin{frame}{Regime Identification via Clustering (1/3)}
\phantomsection\label{regime-identification-via-clustering-13}
\begin{columns}[T,totalwidth=\textwidth]

  % Left: K-Means Figure
  \begin{column}{0.48\textwidth}
    \begin{center}
      \textbf{K-Means (k = 12)} \\[0.3em]
      {\tiny within shares of 0/1} \\[0.3em]
      \includegraphics[width=0.9\textwidth]{pies_kmeans_label_binary_pies_1.png}
    \end{center}
  \end{column}

  % Right: Text
  \begin{column}{0.52\textwidth}
    \textbf{Key Insight}\\[0.6em]
    \begin{itemize}
      \item Many clusters show \textbf{near-pure 0/1 splits} for USREC, ZLB, or COVID dummies.
      \item Clusters align closely with well-known \textbf{macroeconomic regimes}  
            (recession periods, ZLB episodes, COVID-19 shock).
      \item \textbf{Summary}  
            Clearly identifies regimes such as high-inflation periods and recessions.
    \end{itemize}
    {\scriptsize
      \vspace{0.4em}
      *Hierarchical clustering yields similar regime separation.
    }
  \end{column}

\end{columns}
\end{frame}

\begin{frame}{Regime Identification via Clustering (2/3)}
\phantomsection\label{regime-identification-via-clustering-23}
\begin{columns}[T,totalwidth=\textwidth]

  % Left: K-Means Figure
  \begin{column}{0.48\textwidth}
    \begin{center}
      \textbf{K-Means (k = 12)} \\[0.3em]
      {\tiny within shares of 0/1} \\[0.3em]
      \includegraphics[width=0.9\textwidth]{pies_kmeans_label_binary_pies_2.png}
    \end{center}
  \end{column}

  % Right: Text
  \begin{column}{0.52\textwidth}
    \textbf{Key Insight}\\[0.6em]
    \begin{itemize}
      \item Many clusters show \textbf{near-pure 0/1 splits} for USREC, ZLB, or COVID dummies.
      \item Clusters align closely with well-known \textbf{macroeconomic regimes}  
            (recession periods, ZLB episodes, COVID-19 shock).
      \item \textbf{Summary}  
            Clearly identifies regimes such as high-inflation periods and recessions.
    \end{itemize}
    {\scriptsize
      \vspace{0.4em}
      *Hierarchical clustering yields similar regime separation.
    }
  \end{column}

\end{columns}
\end{frame}

\begin{frame}{Regime Identification via Clustering (3/3)}
\phantomsection\label{regime-identification-via-clustering-33}
\begin{columns}[T,totalwidth=\textwidth]

  % Left: K-Means Figure
  \begin{column}{0.48\textwidth}
    \begin{center}
      \textbf{K-Means (k = 12)} \\[0.3em]
      {\tiny within shares of 0/1} \\[0.3em]
      \includegraphics[width=0.9\textwidth]{pies_kmeans_label_binary_pies_3.png}
    \end{center}
  \end{column}

  % Right: Text
  \begin{column}{0.52\textwidth}
    \textbf{Key Insight}\\[0.6em]
    \begin{itemize}
      \item Many clusters show \textbf{near-pure 0/1 splits} for USREC, ZLB, or COVID dummies.
      \item Clusters align closely with well-known \textbf{macroeconomic regimes}  
            (recession periods, ZLB episodes, COVID-19 shock).
      \item \textbf{Summary}  
            Clearly identifies regimes such as high-inflation periods and recessions.
    \end{itemize}
    {\scriptsize
      \vspace{0.4em}
      *Hierarchical clustering yields similar regime separation.
    }
  \end{column}

\end{columns}
\end{frame}

\begin{frame}{ML Model Comparison: Accuracy}
\phantomsection\label{ml-model-comparison-accuracy}
\begin{columns}[T, totalwidth=\textwidth]

% ---------- Left Column: Tables ----------
\begin{column}{0.60\textwidth}

% --- Table 1: Accuracy Comparison ---
\scriptsize
\begin{center}
\begin{tabularx}{0.90\textwidth}{llc}
\toprule
Task & Model & Accuracy \\
\midrule
Unemployment & Naive Bayes & 0.86 \\
Unemployment & Decision Tree & 0.97 \\
Unemployment & XGBoost-A & 0.96 \\
Unemployment & XGBoost-B & 0.96 \\
Unemployment & XGBoost-C & 0.96 \\
\midrule
Inflation (YoY) & Naive Bayes & 0.86 \\
Inflation (YoY) & Decision Tree & 0.93 \\
Inflation (YoY) & XGBoost-A & 0.84 \\
Inflation (YoY) & XGBoost-B & 0.83 \\
Inflation (YoY) & XGBoost-C & 0.86 \\
\midrule
Inflation (MoM) & Naive Bayes & 0.71 \\
Inflation (MoM) & Decision Tree & 0.65 \\
Inflation (MoM) & XGBoost-A & 0.57 \\
Inflation (MoM) & XGBoost-B & 0.64 \\
Inflation (MoM) & XGBoost-C & 0.60 \\
\bottomrule
\end{tabularx}

\vspace{0.4em}
\raggedright\tiny
Notes: XGBoost-A is a baseline configuration with moderate depth and shrinkage. XGBoost-B has deeper trees and more estimators, optimized for capturing nonlinearities. XGBoost-C is shallow but high-learning-rate model emphasizing speed and simplicity. Please refer to the appendix for detailed parameters.

\end{center}
\normalsize

\end{column}

% ---------- Right Column: Text ----------
\begin{column}{0.40\textwidth}

\textbf{Key Insights}\\[0.4em]

\begin{itemize}
  \small
  \item Decision Tree and XGBoost achieve \textbf{very high accuracy} for unemployment classification.
  \item Inflation (YoY) is predicted well by all methods, with Decision Tree performing best.
  \item Inflation (MoM) remains the most challenging task across models.
  \item XGBoost parameter variations show clear trade-offs between model complexity and generalization.
\end{itemize}

\end{column}

\end{columns}
\end{frame}

\section{Conclusion}\label{conclusion}

\begin{frame}{Conclusion, Implications, and Future Work}
\phantomsection\label{conclusion-implications-and-future-work}
\begin{block}{Conclusion}
\phantomsection\label{conclusion-1}
\begin{itemize}
\tightlist
\item
  PCA reveals distinct inflation and business-cycle factors.
\item
  Clustering effectively identifies major macroeconomic regimes.
\item
  ML models classify unemployment and YoY inflation well; MoM inflation
  is noisy.
\end{itemize}
\end{block}

\begin{block}{Implications}
\phantomsection\label{implications}
\begin{itemize}
\tightlist
\item
  Macroeconomic dynamics are multi-factor and complex.
\item
  ML can support rapid macro-state monitoring.
\item
  Short-term inflation measures require cautious interpretation.
\end{itemize}
\end{block}

\begin{block}{Future Work}
\phantomsection\label{future-work}
\begin{itemize}
\tightlist
\item
  Add high-frequency data and extend models to other methods.
\item
  Identify regime-specific shifts in inflation--unemployment dynamics.
\item
  Comparison and integration with traditional macroeconomic models.
\end{itemize}
\end{block}
\end{frame}

\begin{frame}
\begin{center}
\vspace{1cm}
{\huge Thank you!}
\end{center}
\end{frame}

\begin{frame}
\begin{center}
\vspace{1cm}
{\huge Appendix}
\end{center}
\end{frame}

\begin{frame}{XGBoost Parameter Summary}
\phantomsection\label{xgboost-parameter-summary}
\begin{center}
\tiny
\begin{tabularx}{0.80\textwidth}{lccccc}
\toprule
Model & n\_estimators & max\_depth & learning\_rate & subsample & colsample\_bytree \\
\midrule
XGBoost-A & 200 & 3 & 0.05 & 0.8 & 0.8 \\
XGBoost-B & 300 & 4 & 0.03 & 0.7 & 0.9 \\
XGBoost-C & 150 & 2 & 0.10 & 0.9 & 0.7 \\
\bottomrule
\end{tabularx}
\end{center}
\normalsize
\end{frame}

\begin{frame}
\begin{center}
\vspace{1cm}
{\huge References}
\end{center}
\end{frame}

\begin{frame}[allowframebreaks]{}
\phantomsection\label{section}
\phantomsection\label{refs}
\begin{CSLReferences}{1}{0}
\bibitem[\citeproctext]{ref-af24a568-7833-3e7d-8cb0-64a4701160e5}
Blanchard, Olivier. 2016. {``The Phillips Curve: Back to the '60s?''}
\emph{The American Economic Review} 106 (5): 31--34.
\url{http://www.jstor.org/stable/43860981}.

\bibitem[\citeproctext]{ref-Calvo1983StaggeredPrices}
Calvo, Guillermo A. 1983. {``Staggered Prices in a Utility-Maximizing
Framework.''} \emph{Journal of Monetary Economics} 12 (3): 383--98.
\url{https://doi.org/10.1016/0304-3932(83)90060-0}.

\bibitem[\citeproctext]{ref-Gogas2022}
Gogas, Periklis, Theophilos Papadimitriou, and Emmanouil Sofianos. 2022.
{``Forecasting Unemployment in the Euro Area with Machine Learning.''}
\emph{Journal of Forecasting} 41 (3): 551--66.
https://doi.org/\url{https://doi.org/10.1002/for.2824}.

\bibitem[\citeproctext]{ref-Hazell2022PhillipsCurve}
Hazell, Jonathon, Juan Herreño, Emi Nakamura, and Jón Steinsson. 2022.
{``The Slope of the Phillips Curve: Evidence from u.s. States.''}
\emph{The Quarterly Journal of Economics} 137 (3): 1299--1344.
\url{https://doi.org/10.1093/qje/qjac010}.

\bibitem[\citeproctext]{ref-LUCAS197619}
Lucas, Robert E. 1976. {``Econometric Policy Evaluation: A Critique.''}
\emph{Carnegie-Rochester Conference Series on Public Policy} 1: 19--46.
https://doi.org/\url{https://doi.org/10.1016/S0167-2231(76)80003-6}.

\bibitem[\citeproctext]{ref-MAGAZZINO2025471}
Magazzino, Cosimo, Marco Mele, and Mihai Mutascu. 2025. {``An Artificial
Neural Network Experiment on the Prediction of the Unemployment Rate.''}
\emph{Journal of Policy Modeling} 47 (3): 471--91.
https://doi.org/\url{https://doi.org/10.1016/j.jpolmod.2024.10.004}.

\bibitem[\citeproctext]{ref-InflationDynamicsandtheGreatRecession}
Mazumder, Sandeep, and Laurence M. Ball. 2011. {``Inflation Dynamics and
the Great Recession.''} \emph{IMF Working Papers} 2011 (121): A001.
\url{https://doi.org/10.5089/9781455263387.001.A001}.

\bibitem[\citeproctext]{ref-Medeiros02012021}
Medeiros, Marcelo C., Gabriel F. R. Vasconcelos, Álvaro Veiga, and
Eduardo Zilberman. 2021. {``Forecasting Inflation in a Data-Rich
Environment: The Benefits of Machine Learning Methods.''} \emph{Journal
of Business \& Economic Statistics} 39 (1): 98--119.
\url{https://doi.org/10.1080/07350015.2019.1637745}.

\bibitem[\citeproctext]{ref-af978182-58ee-3c82-839b-4d1eba4da5d0}
Muth, John F. 1961. {``Rational Expectations and the Theory of Price
Movements.''} \emph{Econometrica} 29 (3): 315--35.
\url{http://www.jstor.org/stable/1909635}.

\bibitem[\citeproctext]{ref-53cde4ea-d643-31bc-af9d-f3aefdafb5f0}
Phillips, A. W. 1958. {``The Relation Between Unemployment and the Rate
of Change of Money Wage Rates in the United Kingdom, 1861-1957.''}
\emph{Economica} 25 (100): 283--99.
\url{http://www.jstor.org/stable/2550759}.

\bibitem[\citeproctext]{ref-doi:10.1086ux2f261117}
Rotemberg, Julio J. 1982. {``Sticky Prices in the United States.''}
\emph{Journal of Political Economy} 90 (6): 1187--1211.
\url{https://doi.org/10.1086/261117}.

\bibitem[\citeproctext]{ref-90faeace-9ccb-36dc-bd27-e5d7bec510cf}
Samuelson, Paul A., and Robert M. Solow. 1960. {``Analytical Aspects of
Anti-Inflation Policy.''} \emph{The American Economic Review} 50 (2):
177--94. \url{http://www.jstor.org/stable/1815021}.

\bibitem[\citeproctext]{ref-bbf94ffe-9638-3ab0-8e92-b9f6f5633382}
Sargent, Thomas J., and Neil Wallace. 1975. {``"Rational" Expectations,
the Optimal Monetary Instrument, and the Optimal Money Supply Rule.''}
\emph{Journal of Political Economy} 83 (2): 241--54.
\url{http://www.jstor.org/stable/1830921}.

\bibitem[\citeproctext]{ref-NBERw25987}
Stock, James H, and Mark W Watson. 2019. {``Slack and Cyclically
Sensitive Inflation.''} Working Paper 25987. Working Paper Series.
National Bureau of Economic Research.
\url{https://doi.org/10.3386/w25987}.

\bibitem[\citeproctext]{ref-Woodford2003InterestAndPrices}
Woodford, Michael. 2003. \emph{Interest and Prices: Foundations of a
Theory of Monetary Policy}. Princeton, NJ: Princeton University Press.
\url{https://press.princeton.edu/books/hardcover/9780691010496/interest-and-prices}.

\end{CSLReferences}
\end{frame}

\end{document}
